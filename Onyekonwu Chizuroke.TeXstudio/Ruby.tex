\documentclass{article}
\begin{document}
	\title{\underline{\textbf{RUBY}}}
	\maketitle
	\section{WHY RUBY}
	Ruby is a dynamic, open source programming language with a focus on simplicity and productivity. Ruby is most often used for creating web applications but is useful for many other programs as well. (FROM: ChrismaO, Wikipedia)
	\section{WHAT IS RUBY?}
	Ruby is an interpreted, high level, general purpose programming language. Ruby is dynamically typed and uses garbage collection and just-in-time compilation. It supports multiple programming paradigms, including procedural, object oriented and functional programming.
	\section{BRIEF HISTORY OF RUBY (FROM: ChrismaO; Wikipedia)}
	Ruby was created by \emph{Yukihiro Matsumato,} or “MATZ” in Japan in the mid 1990’s. It was designed for programmer productivity with the idea that programming should be fun for programmers. 
	\paragraph{The name “RUBY” originated during an online chat session between Matsumoto and Keiju Ishitsuka on Feb. 24 1993, before any code had been written for the language.}
   \paragraph{The first public release of Ruby 0.95 was announced on December 21, 1995.} Following this release, several stable versions of Ruby were released in the following years.
   \begin{itemize}
   \item Ruby 1.0: Dec 25, 1996                           
	\item Ruby 1.2 : Dec 1998
	\item Ruby 1.4: August 1999
	\item Ruby 1.6 : September 2000
	\end{itemize}
\end{document}